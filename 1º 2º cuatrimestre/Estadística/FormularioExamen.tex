\documentclass[12pt,a4paper]{report}
\usepackage[utf8]{inputenc}
\usepackage{amsmath}
\usepackage{amsfonts}
\usepackage{amssymb}
\usepackage{graphicx}
\usepackage[left=2cm,right=2cm,top=2cm,bottom=2cm]{geometry}
\author{Cristina Sánchez Justicia }
\title{Apuntes mínimos examen final Estadística}
\begin{document}
\section*{Estadística: Apuntes para el examen final}
\subsection*{1. Estadística descriptiva unidimensional}

Tabla: $x_i  n_i  N_i  f_i  F_i  a_i  h_i$
\newline
$a_i = e_i - e_{i-1}$
\newline
$h_i = \frac{n_i}{a_i}$
\\\\
\\
\\
\textbf{Variables discretas: }

Mediana: \[ \exists F_i = 0.5 \longrightarrow Me = \frac { x_i + x_{i+1}}{2} \]
\newline
\newline
\newline
\textbf{Variables continuas: }

Mediana: \[ \nexists F_i = 0.5 , F_i > 0.5 \longrightarrow Me = e_{i-1} + \frac {0.5 - F_{i-1}}{f_i} a_i \]

Notación intervalo percentil 74 $ I_{P_{74}}$

Moda: \[ max h_i \longrightarrow Mo = e_{i-1} + \frac {h_i - h_{i-1}}{ (h_i - h_{i-1}) + (h_i - h_{i+1}) } a_i\]
\newline
\newline
\newline
\textbf{Común: }

Varianza:  \[ \sigma ^ 2 = \frac {\sum x_i ^ 2 n_i}{N} - \overline{x}^2 \]

Tabla para calcular la varianza: $ x_i n_i x_in_i x^2_in_i $\\


Desviación típica: si es más baja, la población está mejor
representada por la media
\newline

Coeficiente de variación: \[ CV_x = \frac {\sigma _ x} {\overline x} \]

Sirve para comparar muestras. Mientras menor es, más homogénea es la muestra.
\newpage

             \subsection*{2. Estadística descriptiva bidimensional }

\textbf{Covarianza: } \[ \sigma_{xy} =
\frac { \sum _ {i} { \sum _ {j} { x_i y_j n_{ij} } }}{N} - \overline x \overline y \]

Como debe quedar en la calculadora: $ \sum xy \div n - \overline x x \overline y $
\newline
\newline
Significado de $\sigma _ {xy} $:

- Positivo: Las variables varían en el mismo sentido

- Negativo: Las variables varían en sentidos opuestos

- Cero: La relación no es lineal
\newline
\newline
\textbf{Recta de regresión lineal: }
\[Y = f(X) = a + bx \]
\[ b = \frac {\sigma _{xy}}{\sigma _ x ^ 2} , a = \overline y - b \overline x \]
\textbf{Coeficiente de correlación lineal: }
\newline
$ -1 \leqslant r \leqslant 1 $
\[ r = \frac {\sigma _ {xy}}{\sigma _ x \sigma _ y} = \sqrt{b b'}\]
La relación lineal es mayor cuando $|r| \simeq 1$
\newline
\textbf{Coeficiente de determinación: }
\[ R ^ 2 = 1 - \frac{\sigma _ {res} ^ 2}{\sigma _ y ^ 2}\]
\[ \sigma _ {res} ^ 2 = \sigma _ y ^ 2 (1 - r ^ 2 ) \]
Posición de las rectas:

$r \neq 1$ Las rectas se cortan en $(\overline x , \overline y )$

r = 1 Las rectas coinciden

r = 0 No hay relación entre las variables

\newpage
\subsection*{3. Probabilidad}
\textbf{Teorema de la probabilidad total: }
\[ P(B) = \sum P(A_i) P(B|A_i) \]
\textbf { Teorema de Bayes: }

H = hipótesis
E = evidencia

\[ P(H|E) = \frac {P(H) P(E|H)}{P(E)} = \frac {P(H \cap E)} {P(E)} \]

\newpage
\subsection*{4. Variables aleatorias }
\textbf{Discreta: }
\newline

$E(X) = \sum x_i p_i $
\newline

$Var(X) = E(X ^ 2 ) - E (X) ^ 2$
\newline
\newline
\textbf{Continua: }
\newline

Función de densidad:
\[ P(x_i < X < x _ j) = \int _ {x _ i} ^ {x _ j} f(x) dx  \]

Función de distribución:
\[ F(x) = \int _ { - \infty} ^ x f(t)dt\]

Media y varianza:
\[ E(x) = \int _ {- \infty} ^ {\infty} x f(x)dx , Var (x) = E(x^ 2) - E(x) ^ 2\]

\newpage
\subsection*{5. Modelos de distribuciones }
\textbf{Binomial: } $ X \rightsquigarrow B(n, p) $

\[ P [X = K ] = \binom {n}{k} p ^k (1 - p) ^ {n - k} \]

$ E[x] = np $
\newline

$ Var[x] = np (1 - p ) $
\newline
\newline
\newline
\newline
\textbf{Poisson: } $X \rightsquigarrow P (\lambda) $

\[ P[ X = k ] = e ^ {-\lambda} \frac{\lambda ^ k }{ K ! } \]

$ E [X] = Var [X] = \lambda $
\newline
\newline
\newline
\textbf{Normal: } $ X \rightsquigarrow N (\mu, \sigma) $
\newline

$ Z = \frac{X - \mu}{\sigma} $
                                                 \newline
\newline
\newline
\newline
\textbf{APROXIMACIONES: }
\newline

\textbf{B (n,p) $\approx$ P ( np)} si $ n \geqslant 30$ , $ p < 0.1$ y $ np \leqslant 5$
\newline

\textbf{B(n,p $\approx $ N (np, $\sqrt{np(1-p)}$)} si $ n \geqslant 30 $, $  np > 5$ y $p \in (0.1 , 0.9) $ 
\newline

\textbf{P($\lambda$) $\approx $ N ($\lambda, \lambda$)} si $\lambda \geqslant 10 $ 
\\\\
\\
\textbf{Corrección por continuidad: }
\[ P [x = k ] = P [k - 0'5 \leqslant X \leqslant k + 0'5] \]

\newpage
\subsection*{6, 7 y 8. Contrastes de hipótesis }
\textbf{Cuasivarianza muestral: }
\[ s ^ 2 = \frac{n}{n - 1} \sigma ^ 2 \]
\\
\\
El resto está en los formularios

\newpage
\subsection*{9. Optimización sin restricciones }
\subsubsection{Gradiente }
\[ \bigtriangledown f(x,y) = (\frac{d f(x, y)}{dx},\frac{df(x,y)}{dy}) \] 
\subsubsection{Puntos críticos (a,b) }
\[ \bigtriangledown f(x,y) = 0 \]
Resolver sistema de ecuaciones
\subsubsection{Matriz Hessiana }
\[ H [f(x,y)] =  
\left[ {\begin{array}{cc}
\frac{d^2f(x,y)}{dx^2} & \frac{d^2f(x,y)}{dxdy} \\
\frac{d^2f(x,y)}{dxdy} & \frac{d^2f(x,y)}{dy^2} \\
\end{array} } \right]
\]
\subsubsection{Menores principales}
\[ H _ {1x1} = \frac{d^2 f(x,y)}{dx^2}\]
\[ H _ {2x2} = |H [f(x,y)]|\]

\subsubsection{Analizar signo en (a,b)}
- Si $H _ {1x1} > 0 $ y $H _ {2x2} \geqslant 0$ H es definida positiva (Mínimo) 
\\
- Si $H _ {1x1} < 0 $ y $H _ {2x2} \geqslant 0$ H es definida negativa (Máximo) 
\\
- Si $H _ {1x1} = 0 $ y $H _ {2x2} \geqslant 0$ punto dudoso
\\
- Si $ H _ {2x2} < 0 $ H es indefinida (Punto de silla) 
Representar las funciones en www.wolframalpha.com

\vspace*{\fill}
Si veis algún error, mi telegram es @cristinasj 
\end{document}

                                                                     1,1           Top
